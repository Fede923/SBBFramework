    - 1) [main] SBB + poker integration (ie. inputs, formulas, heuristics (hand balance, reward function...), opponents, opponent model, points, match preprocessing, diversities (+inputs for the diversities)) + current results (types of behaviors + score against baseline opponents (ie. loose aggressive, loose passive, tight aggressive, tight passive))
    2) [extra] Comparison across various diversity metrics and how they impacted the teams' evolution
    3) [extra] Second layer

- definir estrutura para capitulo da literature review da thesis + quais os papers principais
    - Start with what has been done previously with ML and poker, then look at GP and diversity and finish up with SBB and second layer. All this material will find its way into your thesis, and some will be useful for paper writing.
---
Abstract:
[...]

Acknowledgements:
[...]

Contents:
[...]

1. Introduction (+- 2 pages)
    [quick overview of ML and poker]
    [quick overview of SBB]
    [thesis goals]
    [thesis contributions]

(Literature Review)
2. Machine Learning and Poker
    1. Poker
        1. Introduction
            [overview of poker] [13]
            [poker as a testbed for AI] [13]
            [overview of types of poker ([7][8] como One Card Poker, [9] draw poker), limit/no limit, number of players)
            [poker researches] ([7])
            [explain choice of poker type for this research] [13]
        2. Texas Hold’em Poker [6][12]
            [explain the rules, give exmaple of researches that used it]
        3. Inputs/Formulas [6][13]
            [HS, potential, EHS, pot odds, aggressiveness...]
        4. Playing Styles [4][5][7][8]
            [agressive/passive, tight/loose]
        5. Biasing the hand types [4]
            [why is it necessary, ways to do it]
        6. Exploitative and Nash Equilibrium Learning Strategies [13]
            [explain each one and what are their goals]
    2. Machine Learning
        1. Introduction
            [overview of machine learning]
        2. Reinforcement Learning
            1. Introduction
                [overview of reinforcement learning]
                [examples of applications]
            2. Opponents
                1. Self-play [6]?[11][12]
                2. Coevolution [2][6][12]
                    [competitive and cooperative coevolutions]
                3. Static/Fixed opponents [4][5][6][12]
            3. Opponent Model [6][7][8][10]
                [why it is necessary]
                [strategies for opponent modeling]
            4. Hall of Fame [2][6]
                [how it works, why it is necessary]
                [talk about Evolutionary Forgetting]
        3. Genetic Programming
            1. Introduction
                [explain what it is, its applications, and advantages/disadvantages]
            2. Diversity Maintenance
                1. Introduction [1]
                    [why is it necessary]
                    [based on genotype and based on phenotype]
                2. Genotype Distance (eg. euclidean) [17]
                3. Hamming distance [1]
                4. Euclidean distance [1]
                5. Fitness Sharing [2]
                6. NCD [1]
                7. Entropy [1]
            3. Pareto [11][12]
                [what it is]
                [can be used to select distictions]
                [can be used to balance fitness and diversity]

    3. Previous Works
        x. +3[6] [select and explain some interesting papers about machine learning and poker + compare with this research]

3. Symbiotic Bid-based GP (SBB)
    1. Introduction
        [what it is and how it works] [14][15][16][more details: 18]
        [cite in what other domains SBB has been used] [15][16][17][18]
    2. Architecture: [initialization, representation, selection, replacement, execution, variation operations, details of how the coevolution works and the architecture] [14][15][16]
        1. Symbiont Population
        2. Host Population
        3. Point Population        
    3. Hierarchical SBB [15][16]
        [why use a second layer]
        [how it is implemented]

(My research)
4. SBB for Poker
[...] (what was implemented, how it works, explain the choices made, inputs[6][10][12], opponent model[6][7][8][10], opponents[4][5][7][8], diversities(+coding, quantization, groups)[1], hand sampling/balancing strategy[4], point population, reward function, pareto, training/validation/champion populations, hall of fame criteria[2][6], saving results as .json files(highly compatible with other systems), unlimited chips due to ACPC...)
+ a diferenca from the previous SBB implementations: if works as an if (not as a change signal)

5. Experiments and Results
[...] (methodology, parametrization, results...)

6. Conclusions, Contributions and Future Work:
[...]

Appendixes
[...] (parameters)

Bibliography
[...]


50-90 pages

head start?

Used papers:
1. Sustaining Diversity using Behavioral Information Distance, 2009 (Faustino J. Gomez) [DONE, A] (diversities, determinitic crowding)
2. New Methods for Competitive Coevolution, 1996 [DONE, B] (fitness sharing, hall of fame, competitive coevolution)
3. Heads-up limit hold’em poker is solved, 2015 [DONE, C] (university of alberta solution for poker)
4. Evolving Adaptive Play for Simplified Poker, 1998 [DONE, A] (poker, the four types of strategies/opponents, balance of hand types)
5. An Adaptive Learning Model for Simplified Poker Using Evolutionary Algorithms, 1999 [DONE, C] (poker, the four types of strategies/opponents, coded opponent)
6. Countering Evolutionary Forgetting in No-Limit Texas Hold’em Poker Agents, 2012 [DONE, A] (texas holdem, hall of fame, coevolution, inputs)
7. Bayesian Opponent Modeling in a Simple Poker Environment, 2007 [DONE, A] (poker, the four types of strategies/opponents, defining opponents by alfa/beta, input: last opponent action, opponent model, features for a world-class poker player, good overview of poker in ML)
8. Can Opponent Models Aid Poker Player Evolution?, 2008 [DONE, A] (continuacao do [7], com os mesmos topicos)
9. An Investigation of an Adaptive Poker Player, 2001 [DONE, B] (poker, four static opponents(based on rules), draw poker)
10. Learning Strategies for Opponent Modeling in Poker, 2013 [DONE, A] (poker, opponent modeling, inputs)
11. Pareto coevolution: Using performance against coevolved opponents in a game as dimensions for Pareto selection, 2001 [DONE, A] (poker, self-play, pareto)
12. Finding robust texas hold'em poker strategies using pareto coevolution and deterministic crowding, 2002 [DONE, A] (poker, pareto, coevolution, determinitic crowding, inputs, head start, self-play, fixed opponents, poker holdem, more than 2 players)
13. Computer poker: A review, 2011 [DONE, A] (poker, formulas, nash x exploration, poker+ML overview)
14. Symbiosis, Complexication and Simplicity under GP, 2010 [DONE, A] (sbb, sbb overview, crossover vs mutation)
15. Hierarchical Task Decomposition through Symbiosis in Reinforcement Learning, 2012 [DONE, A] (sbb, sbb overview, task: Acrobot, sbb second layer)
16. On Run Time Libraries and Hierarchical Symbiosis, 2012 [DONE, A] (sbb, rtl, sbb overview, task: Pinball, sbb second layer)
17. Genotypic versus Behavioural Diversity for Teams of Programs Under the 4-v-3 Keepaway Soccer Task, 2014 [DONE, A] (sbb, task: keepaway soccer, genotype diversity)
18. A Symbiotic Bid-Based Framework for Problem Decomposition using Genetic Programming, 2011 [DONE, A] (sbb, sbb details, sbb overview, sbb main paper, application: classification and temporal sequence learning, conferir para ver mais secoes que podem ser usados em GP, ou coevolution)


Not used papers:
- An evolutionary game-theoretic analysis of poker strategies, 2009 [DONE, B] (poker, four static opponents, VPIP and AGR)


Papers to give a look:

poker opponent model:
- Billings, D. Papp, D. Schaeffer, J. Szafron, D Opponent modeling in
poker Proceedings of the fifteenth nationalltenth conference on
Artificial intelligence/Innovative applications of artificial
intelligence, 1998, (pp. 493 - 499)
- Southey, F. Bowling, M.P. Larson, B. Piccione, C. Burch, N. Billings,
D. Rayner, Bayes' Bluff: Opponent Modelling in Poker. In
Proceedings of the 21st Annual Conference on Uncertainty in
Artificial Intelligence (UAI-05), 2005, (pp 550-555)
- Schauenberg, T.: Opponent modelling and search in poker. Master’s thesis. University of
Alberta (2006) x2
- A Davidson, (1999) Using Artificial Neural Networks to Model
Opponents in Texas Hold 'Em. [Unpublished manuscript]. Available:
http://spaz.ca/aaronJpoker/nnpoker.pdf.
- A Davidson, D. Billings, J. Schaeffer, and D. Szafron, Improved
Opponent Modeling in Poker. Proceedings of the 2000 International
Conference on Artificial Intelligence (lCA/'2000). 1999,1467--1473.
- Felix, D., and Reis, L. 2008. An experimental approach to
online opponent modeling in texas hold’em poker. Advances
in Artificial Intelligence-SBIA 2008 83–92.
- Ponsen, M.; Ramon, J.; Croonenborghs, T.; Driessens, K.;
and Tuyls, K. 2008. Bayes-relational learning of opponent
models from incomplete information in no-limit poker. In
Twenty-third Conference of the Association for the Advancement
of Artificial Intelligence (AAAI-08).
- Southey, F.; Bowling, M.; Larson, B.; Piccione, C.; Burch,
N.; Billings, D.; and Rayner, C. 2005. Bayes bluff: Opponent
modelling in poker. In In Proceedings of the 21st
Annual Conference on Uncertainty in Artificial Intelligence.
- M. Ponsen, J. Ramon, T. Croonenborghs, K. Driessens, K. Tuyls. Bayes-relational
learning of opponent models from incomplete information in no-limit poker,
in: 23rd Conference of the Association for the Advancement of Artificial
Intelligence (AAAI-08), Chicago, USA, 2008, pp. 1485–1487.
- F. Southey, M. Bowling, B. Larson, C. Piccione, N. Burch, D. Billings, D.C. Rayner.
Bayes’ bluff: opponent modelling in poker, in: Proceedings of the 21st
Conference in Uncertainty in Artificial Intelligence (UAI ’05), 2005, pp. 550–
558.

poker+ML overview:
- * Schaeffer, J. Billings, D. Pefia, L. Szafron, D. Learning to play strong
poker In ICMLA-99, Proceedings of the 16th International Conference
on Machine Learning, 1999
- D. Sklansky, The Theory of Poker. Two Plus Two Publishing, 1992.
- N. Findler, Studies in Machine Cognition Using the Game of Poker.
CACM 20(4), pp 230-245, 1977
- Burns, K. Style in poker, In Proceedings of2006 IEEE Symposium on
Computational Intelligence and Games (CIG), (pp.257-264)
- Billings, D., Davidson, A., Schaeffer, J., Szafron, D.: The challenge of poker. Artificial Intelligence
134, 201–240 (2002) x2
- Billings, D.: Algorithms and Assessment in Computer Poker. PHD Dissertation. University
of Alberta (2006)
- Korb, K.; Nicholson, A.; and Jitnah, N. 1999. Bayesian
poker. In Proceedings of the Fifteenth Conference on Uncertainty
in Artificial Intelligence, 343–350. Morgan Kaufmann
Publishers Inc.
- Barone L. and While L. 1998. Evolving Computer Opponents to Play a Game of Simplified
Poker. In proceedings of the 1998 International Conference on Evolutionary Computation
(ICEC’98), pp 108-113
- Barone L. and While L. 2000. Adaptive Learning for Poker. In proceedings of the Genetic and
Evolutionary Computation Conference 2000 (GECCO’2000), July 10-12, Las Vegas,
Nevada, pp 560-573
- Billings, D., Papp, D., Schaeffer, J. and Szafron, D. 1998a. Poker as a Testbed for AI Research.
In Proceedings of AI’98, The Twelfth Canadian Conference on Artificial Intelligence,
Mercer, R.E., and Neufeld, E. (eds), Advances in Artificial Intelligence, Springer-Verlag, pp
228-238
- Billings, D., Pe•a, L, P., Schaeffer, J. and Szafron, D. 1999. Using Probabilistic Knowledge
and Simulation to Play Poker. In Proceedings of AAAI-99 (Sixteenth National Conference
of the Association for Artificial Intelligence).
- Findler N. 1977. Studies in machine cognition using the game of poker. Communications of
the ACM, 20(4):230-245.
- Schaeffer, J., Billings, D., Pe•a, L, P. and Szafron, D. 1999. Learning to Play Strong Poker. In
Proceedings of the Sixteenth International Conference on Machine Learning (ICML-99)
(invited paper)
- Beattie, B., Nicolai, G., Gerhard, D., Hilderman, R.: Pattern classification in No-Limit Poker:
A head start evolutionary approach. In: Canadian Conference on AI, pp. 204–215 (2007)
- Booker, L.: A No Limit Texas Hold’em poker playing agent. Master’s Thesis. University of
London (2004)
- Johanson, M.: Robust strategies and counter-strategies: Building a champion level computer
poker player. Master’s thesis. University of Alberta (2007)
- Billings, D. Burch, N. Davidson, A. Holte, R. Schaeffer, J. Schauenberg
T, and Szafron, D. Approximating game-theoretic optimal strategies for
full-scale poker In Proceedings of the eighteenth International Joint
Conference on Artificial Intelligence 2003, (pp. 661-668).
- Schaeffer, J. Billings, D. Pefia, L. Szafron, D. Learning to play strong
poker In ICMLA-99, Proceedings of the 16th International Conference
on Machine Learning, 1999
- E. Saund, Capturing the infonnation conveyed by opponents' betting
behaviour in poker. In proceedings of 2006 IEEE Symposium on
Computational Intelligence and Games (CIG), (pp. 126-133)

poker only:
- Sklansky, D. 1994. Theory of Poker. Two Plus Two Publishing, ISBN 1-880685-00-0
- Sklansky, D. 1996. Hold'em Poker. Two Plus Two Publishing, ISBN 1-880685-08-6

ML overview:
- Alpaydin, E. 2010. Introduction to Machine Learning. The
MIT Press.

diversity:
- Cuccu, G., and Gomez, F. 2011. When novelty is not
enough. In EvoApplications – Part I, volume 6624 of LNCS.
- Gomez, F. 2009. Sustaining diversity using behavioral information
distance. In ACM Conference on Genetic and Evolutionary
Computation.

GP overview:
- A. Teller. Advances in Genetic Programming,
chapter 9. MIT Press, 1994.
- J. R. Koza. Genetic Programming: On the programming of computers by
means of natural selection. MIT, 1992.
- A. R. McIntyre and M. I. Heywood, “Classification as clustering: A
Pareto cooperative-competitive GP approach,” Evolutionary Computation,
vol. 19, no. 1, pp. 137–166, 2011.
- I. Kushchu, “Genetic programming and evolutionary generalization,”
IEEE Transactions on Evolutionary Computation, vol. 6, no. 5, pp. 431–
442, 2002.

RL:
- L. P. Kaelbling, M. L. Littman, and A. W. Moore.
Reinforcement learning: A survey. Journal of Machine
Learning Research, 4:237–285, 1996.
- D. Moriarty, A. C. Schultz, and J. J. Grefenstette.
Evolutionary algorithms for reinforcement learning. J.
of Machine Learning Res., 11:241–276, 1999.

pareto:
- E. D. de Jong. A monolithic archive for Pareto-coevolution. Evolutionary
Computation, 15(1):61{94, 2007.
- M. Lemczyk and M. I. Heywood. Training binary GP classifiers eficiently:
A Pareto-coevolutionary approach. In European Conference on Genetic
Programming, volume 4445 of LNCS, pages 229{240, 2007.
- A. R. McIntyre and M. I. Heywood. Pareto cooperative-competitive Genetic
Programming: A classification benchmarking study. In Genetic Program-
ming Theory and Practice VI, pages 43{60. Springer, 2008.

sbb:
- M. I. Heywood and P. Lichodzijewski. Symbiogenesis as a mechanism for
building complex adaptive systems: A review. In EvoApplications, volume
6024 of LNCS, pages 51{60, 2010.
- P. Lichodzijewski and M. I. Heywood. Managing team-based problem solv-
ing with symbiotic bid-based genetic programming. In Proceedings of the
Genetic and Evolutionary Computation Conference, pages 363{370, 2008.
- P. Lichodzijewski. A symbiotic bid-based framework 
for problem decomposition using Genetic 
Programming. PhD thesis, Faculty of Computer 
Science, Dalhousie University, 2011. 
http://www.cs.dal.ca/ ̃mheywood/Thesis/PhD.html. 
- P. Lichodzijewski, J. A. Doucette, and M. I. Heywood. 
A symbiotic framework for hierarchical policy search. 
Technical Report CS-2011-06, Dalhousie University, 
Faculty of Computer Science, 2011. 
http://www.cs.dal.ca/research/techreports/. 
- P. Lichodzijewski and M. I. Heywood. 
Pareto-coevolutionary Genetic Programming for 
problem decomposition in multi-class classification. In 
Proceedings of the ACM Genetic and Evolutionary 
Computation Conference, pages 464–471, 2007. 
- P. Lichodzijewski and M. I. Heywood. Symbiosis, 
complexification and simplicity under GP. In 
Proceedings of the ACM Genetic and Evolutionary 
Computation Conference, pages 853–860, 2010.
- ——, “Managing team-based problem solving with symbiotic bidbased
Genetic Programming,” in Proceedings of the ACM Genetic and
Evolutionary Computation Conference, 2008, pp. 363–370.
- P. Lichodzijewski and M. I. Heywood, “Pareto-coevolutionary Genetic
Programming for problem decomposition in multi-class classification,”
in ACM Genetic and Evolutionary Computation Conference, 2007, pp.
464–471.

- conferir least_regreat, usado pela University of Alberta?

- why all diversities start at around the same value? (same seed) Why there is absolutelly no diversity when no diversity metric is used? (check if it keeps like this even for multiple runs)
- permitir que no second layer, existe a chance aleatoria da action ser atomica? so even if the teams see new things, they are still able to adapt to them easily? (eg.: new opponent styles, hall of fame) and use the appropriete inputs accordindly?


hall of fame
- not only to avoid evolutionary forgetting, but also to provide opponent that use opponent models themselves, so that the inputs that enable unpredictability can be useful and used by the teams